\documentclass{article}
\usepackage{booktabs, graphicx, hyperref, fontspec}
\usepackage{sectsty}
\allsectionsfont{\sffamily}
\usepackage[margin=1in]{geometry}
\hypersetup{
  colorlinks = true,
  urlcolor = cyan,
 }
 \providecommand{\tightlist}{%
  \setlength{\itemsep}{0pt}\setlength{\parskip}{0pt}}
\newcommand*{\authorfont}{\fontfamily{phv}\selectfont}
\usepackage[]{Fira Sans}

\begin{document}

\sffamily

\centerline{\Huge Economic Development}

\vspace{3 mm}

\centerline{\large Dr.~Ryan Safner}
\vspace{2 mm}
\centerline{\large \href{http://devF21.classes.ryansafner.com}{devF21.classes.ryansafner.com}}

\vspace{5 mm}

\begin{tabular}{@{}p{3.5in}p{3.5in}}           
\textbf{Course}: ECON 317 Fall
2021  & \textbf{Email:}  \href{mailto:safner@hood.edu}{\nolinkurl{safner@hood.edu}}\\
\textbf{Room}: Rosenstock 30 & \textbf{Office:}  Rosenstock 110\\
\textbf{Meets}: TuTh 2:00 PM---3:25PM & \textbf{Hours:}  MW 10:00---11
AM \& by appt\\ 
\end{tabular}

\vspace{5 mm}

\hrule


\begin{quote}
``The consequences for human welfare involved in questions like these
are simply staggering: Once one starts to think about them, it is hard
to think about anything else.'' {--- Robert Lucas (1995 Economics Nobel
Laureate)}
\end{quote}

\textbf{Economic Development} concerns the fundamental question posed by
Adam Smith in 1776: \emph{why are some countries wealthy, and others
poor?}. Before ``economics'' emerged as its own discipline in the
20\textsuperscript{th} century, and adopted more rigorous mathematical
trappings, the exploration of \emph{political economy} attempted to
answer this question: political, social, and cultural institutions and
histories had as much to offer as the division of labor and market
exchanges in answering the challenge. In the aftermath of WWII,
economists in ``the First World'' began more consciously studying how to
promote development in ``the Third World'', while policymakers built
international institutions (the IMF, the World Bank, the Bretton-Woods
financial systems) aimed at securing peace and outwardly developing what
they say as the ``Third World.'' In order to grapple with these key
questions, we will examine a mixture of political economy and economic
history to understand the role of political, cultural, and social
institutions in directing economic activity towards prosperity or
towards ruin. This course, like the professionals dealing with the big
questions, will offer many suggestions but fewer ``correct'' or concrete
answers than you may be used to. You should come to this course as a
willing participant in the ongoing conversation.

The economics of development combines core themes and models of
macroeconomics (growth theory, macroeconomic stability and policy) with
core principles of microeconomics (price theory), and as such, the
\textbf{prerequisites} for this course are \textbf{either ECON 205 or
ECON 206}.

\hypertarget{course-format-and-covid}{%
\section{Course Format (and Covid)}\label{course-format-and-covid}}

As of Fall 2021, all students are expected to be on campus except those
with special approved exemptions. As such, this course will be taught
\textbf{in-person} and \textbf{synchronously} until or unless otherwise
announced.

You are expected to come to class except due to medical reasons or other
legitimate conflicts. Watching videos are not a substitute for attending
class.\footnote{On average, even for students who complete all
  assignments, those that do not regularly attend class suffer by a full
  letter grade,
  (\href{https://www.aeaweb.org/articles?id=10.1257/jep.7.3.167}{Levitt
  1993}).} Please see the \protect\hyperlink{attendance}{attendance
policy} for more.

In any event that we are unable to meet in person, I will hold class
meetings at the same day/time live on Zoom, and post all recorded
lectures via Panopto on Blackboard, and all assignments will be
submitted online (often via Blackboard).

\hypertarget{learning-during-a-global-pandemic}{%
\subsection{Learning During a Global
Pandemic}\label{learning-during-a-global-pandemic}}

While we have made some progress in returning to normal, this remains a
unique semester and a lot of things are still awful right now. None of
us signed up for this. None of us are really okay, we're all just
pretending for everyone else.

Many of you may be dealing with hardships at home and at work, and are
generally juggling many more problems than usual. Everyone's future
plans have been completely put on hold or cancelled to a large degree.

I am prioritizing us supporting each other as human beings during this
crazy era, and will try to use simple, accessible solutions that make
sense for the most people, and above all, to be flexible.

If you tell me you're having trouble, I will do whatever I can to help,
and not judge you or think less of you. I hope you will extend me the
same courtesy.

You never \emph{owe} me personal information about your health (mental
or physical). You are however always welcome to talk to me about things
that you're going through. If I can't help you, I usually know somebody
who can.

I want you to learn a lot from this course, but it is more important for
you to remain healthy, balanced, and grounded during this crisis.

\textbf{I reserve the right to change any part of this syllabus and
course, at my discretion, with proper advance warning.}

\hypertarget{course-objectives}{%
\section{Course objectives}\label{course-objectives}}

{By the end of this course,} you will:

\begin{itemize}
\tightlist
\item
  Explain how the development community measures economic development
\item
  Interpret regression tables in the empirical literature in development
\item
  Demonstrate different theories of economic development
\item
  Explain why various policies aimed at promoting development have
  failed
\item
  Describe essential conditions for successful development
\item
  Discuss the broad economic history of ``the West'', several key
  ``Emerging Markets'' (such as Russia, China, Korea, etc.), and several
  other case studies of developing countries
\end{itemize}

Given these objectives, this course fulfills all three of the learning
outcomes for
\href{https://www.hood.edu/academics/departments/george-b-delaplaine-jr-school-business/student-learning-outcomes}{the
George B. Delaplaine, Jr.~School of Business} Economics B.A. program:

\begin{itemize}
\tightlist
\item
  Use quantitative tools and techniques in the preparation,
  interpretation, analysis and presentation of data and info rmation for
  problem solving and decision making {[}\ldots{]}
\item
  Apply economic reasoning and models to understand and analyze problems
  of public policy {[}\ldots{]}
\item
  Demonstrate effective oral and written communications skills for
  personal and professional success{[}\ldots{]}
\end{itemize}

{Content warning:} this class will cover sensitive political and
cultural topics and compel you to grapple with countries, cultures, and
viewpoints very different from your own. To put it mildly, these topics
may include themes of violence, slavery, imperialism, and different
ideologies inherently wrapped up in the tragic history of both the
developed and developing world.

{Fair warning:} \textbf{Economics is hard.} \emph{This, in particular,
may be of the hardest courses that you will take, primarily due to the
mathematical content.} Using the economic way of thinking is a skill, it
is literally retraining your brain to interpret and analyze the world in
a novel way, and is not something that can be memorized. I will do my
best to make this class intuitive and helpful, if not interesting. If at
any point you find yourself struggling in this course for any reason,
please come see me. Do not suffer in silence! Coming to see me for help
does not diminish my view of you, in fact I will hold you in
\emph{higher} regard for understanding your own needs and taking charge
of your own learning. There are also a some fantastic resources on
campus, such as the
\href{http://www.hood.edu/campus-services/academic-services/index.html}{Center
for Academic Achievement and Retention (CAAR)} and the
\href{http://www.hood.edu/library/}{Beneficial-Hodson Library}.

See my \href{/reference\#tips}{tips for success in this course}.

\hypertarget{required-course-materials}{%
\section{Required Course materials}\label{required-course-materials}}

You can find all course materials at my \textbf{dedicated website} for
this course:
\href{https://devF21.classes.ryansafner.com}{devF21.classes.ryansafner.com}.
Links to the website are posted on our Blackboard course page. Please
familiarize yourself with the website, see that it contains this
\href{https://devF21.classes.ryansafner.com/syllabus/}{syllabus}, guides
for your
\href{https://devF21.classes.ryansafner.com/reference/}{reference}, and
our \href{https://devF21.classes.ryansafner.com/schedule/}{schedule}. On
the schedule page, you can find each module with its own class page
(\textbf{start there!}) along with all related readings, lecture slides,
practice problems, and assignments.

My lecture slides will be shared with you, and serve as your primary
resource, but we have several books that you are required to buy and
read for class discussions. I will discuss more about textbooks and
materials in the first day.

\hypertarget{books}{%
\subsection{Books}\label{books}}

There are two books that we will roughly be following in parallel, both
available at the bookstore (or you can find them on Amazon, ebay, etc).
You may choose to purchase used or old versions, but be aware that
content and ordering may slightly vary across versions.

\begin{enumerate}
\def\labelenumi{\arabic{enumi}.}
\tightlist
\item
  Acemoglu, Daron and James A. Robinson, 2008, \emph{Why Nations Fail:
  The Origins of Power, Prosperity, and Poverty,} New York: Crown
  Business
\item
  Easterly, William, 2000, \emph{The Elusive Quest for Growth:
  Economists' Adventures and Misadventures in the Tropics,} Cambridge,
  Mass: MIT Press
\end{enumerate}

Both books are landmarks in the study of economic development by
renowned development economists and are written for a popular audience.
These books should be easily readable and affordable -- you could buy
and read them at the airport or the beach (should you be nerdy enough
like me). Both are listed as \textbf{required} in the bookstore, but
feel free to get them elsewhere.

The first book is something like ``our textbook'' for the course, as it
outlines many of the key topics that we cover this semester. We will
have frequent readings from it, but my coverage of topics and sequencing
will be different from the book. It is one of my favorite books due to
the central role that different institutions play in determining the
variation among countries today.

The second book is older, but aptly describes the history of development
economics as a field, and is a brilliant and relentless narrative of all
of the policies, fads, and politics of the development community and how
many of them went horribly wrong.

\hypertarget{articles}{%
\subsection{Articles}\label{articles}}

Throughout the course, I will post both required and supplemental
(non-required) readings that enrich your understanding for each topic.
Check \emph{frequently} for announcements and updates to assignments,
readings, and grades.

\hypertarget{assignments-and-grades}{%
\section{Assignments and Grades}\label{assignments-and-grades}}

Your final course grade is the weighted average of the following
assignments. You can find general descriptions for all the assignments
on the
\href{http://devF21.classes.ryansafner.com/assignments/}{assignments
page} and more specific information and examples on each assignment's
page on the
\href{http://devF21.classes.ryansafner.com/schedule/}{schedule page}.

\begin{tabular}{l|l|l}
\hline
Frequency & Assignment & Weight\\
\hline
n & Participation (Average) & 25\%\\
\hline
1 & Country Profile & 5\%\\
\hline
2 & Short Paper & 20\% each\\
\hline
1 & Final Exam & 30\%\\
\hline
\end{tabular}

Each assignment is graded on a 100 point scale. Letter-grade equivalents
are based on the following scale:

\begin{table}
\centering
\begin{tabular}{l|c|l|c}
\hline
Grade & Range & Grade & Range\\
\hline
A & 93–100\% & C & 73–76\%\\
\hline
A− & 90–92\% & C− & 70–72\%\\
\hline
B+ & 87–89\% & D+ & 67–69\%\\
\hline
B & 83–86\% & D & 63–66\%\\
\hline
B− & 80–82\% & D− & 60–62\%\\
\hline
C+ & 77–79\% & F & < 60\%\\
\hline
\end{tabular}
\end{table}

See also my
\href{https://ryansafner.shinyapps.io/dev_grade_calculator/}{
\texttt{Grade\ Calculator}} app where you can calculate your overall
grade using existing assignment grades and forecast ``what if''
scenarios.

These grades are firm cutoffs, but I do of course round upwards
\((\geq\) 0.5) for final grades. A necessary reminder, as an academic, I
am not in the business of \emph{giving} out grades, I merely report the
grade that you \emph{earn}. I will not alter your grade unless you
provide a reasonable argument that I am in error (which does happen from
time to time).

\textbf{No extra credit is available}

\hypertarget{policies-and-expectations}{%
\section{Policies and Expectations}\label{policies-and-expectations}}

This syllabus is a contract between you, the student, and me, your
instructor. It has been carefully and deliberately thought out. (A
syllabus can and will be used as a legal document for disputes tried at
a court of law. Ask me how I know.), and I will uphold my end of the
agreement and expect you to uphold yours.

In the language of game theory, this syllabus is my commitment device. I
am a very understanding person, and I know that exceptions to rules
often need to be made for students. However, to be \emph{fair} to
\emph{all} students the syllabus artificially constrains my ability to
make exceptions at a whim for anyone. This prevents clever students from
exploiting my congenial personality at everyone else's expense. Please
read and familiarize yourself with the course policies and expectations
of you. Chances are, if you have a question, it is answered herein.

\hypertarget{attendence}{%
\subsection{Attendence}\label{attendence}}

Your day-to-day classroom attendance is not graded. My philosophy is
that you are all adults and must take ownership of your own learning or
else you will not succeed. Some assignments may require in-class
participation for credit, and an (unexcused) absence may be detrimental
to your grade. Attending class is one of the strongest predictors of
success.

However, as required under Hood College's ``Promise of Fall Plan,'' (Ch.
2-C) \textbf{your classroom attendance will be recorded at every class
meeting.} This is primarily to facilitate contact tracing.

If you know you will be absent, you are not \emph{required} to let me
know, but it is polite to give notice (Note if I do not reply to an
email of yours letting me know, I am probably busy but will still see it
and appreciate your email). Your absence will be noted and recorded for
the purposes stated above. If, however, we have an assignment due in
class, you \emph{must} notify me ahead of time in order to make
alternate arrangements to still receive credit. Hasty ex-post attempts
to notify me will generate little sympathy.

\hypertarget{late-assignments}{%
\subsection{Late Assignments}\label{late-assignments}}

I will accept late assignments, but will subtract a specified amount of
points as a penalty. Even if it is the last week of the semester, I
encourage you to turn in late work: some points are better than no
points!

I reserve the right to re-weight assignments for students whom I believe
are legitimately unable to complete a particular assignment.

\hypertarget{grading}{%
\subsection{Grading}\label{grading}}

I will try my best to post grades on Blackboard's Grading Center and
return graded assignments to you within about one week of you turning
them in. There will be exceptions. Where applicable, I will post answer
keys once I know most homeworks are turned in (see Late Assignments
above for penalties). Blackboard's Grading Center is the place to look
for your most up-to-date grades. See also my
\href{https://ryansafner.shinyapps.io/dev_grade_calculator/}{
\texttt{Grade\ Calculator}} app where you can calculate your overall
grade using existing assignment grades and forecast ``what if''
scenarios.

\hypertarget{communication-email-slack-and-virtual-office-hours}{%
\subsection{Communication: Email, Slack, and Virtual Office
Hours}\label{communication-email-slack-and-virtual-office-hours}}

Students must regularly monitor their \textbf{Hood email accounts} to
receive important college information, including messages related to
this class. Email through the Blackboard system is my main method of
communicating announcements and deadlines regarding your assignments.
\textbf{Please do not reply to any automated Blackboard emails - I may
not recieve it!}. My Hood email (\texttt{safner@hood.edu}) is the best
means of contacting me. I will do my best to respond within 24 hours. If
I do not reply within 48 hours, do not take it personally, and
\emph{feel free to send a follow up email} in the very likely event that
I genuinely did not see your original message.

Our \href{https://hoodcollegeeconomics.slack.com}{slack channel} is
available to all students and faculty in Economics and Business. I have
invited all of my classes and advisees. It will not be extended to
non-Business/Economics students or faculty. All users must use their
\textbf{hood emails} and \textbf{true first and last names}. Each course
has its own channel, exclusive for verified students in the course, and
myself, by my invite only. As a third party platform, you agree to its
Terms of Service. I have created this space as a way to stay connected,
to help one another, and to foster community. Behaviors such as posting
inappropriate content, harassing others, or engaging in academic
dishonesty, to be determined solely at my discretion, will result in one
warning, the content will be deleted, and subsequent behavior will
result in a ban.

In addition to in-person office hours, you can also make an appointment
for \textbf{``office hours''} on Zoom. You can join in with video,
audio, and/or chat, whichever you feel comfortable with. Of course, if
you are not available during those times, we can schedule our own time
if you prefer this method over email or Slack. If you want to go over
material from class, please have \emph{specific} questions you want help
with. I am not in the business of giving private lectures (particularly
if you missed class without a valid excuse).

Watch the excellent and accurate video explaining office hours
\href{https://gamef21.classes.ryansafner.com/syllabus/\#communication-email-slack-and-virtual-office-hours}{on
the website}.

\hypertarget{netiquette}{%
\subsection{Netiquette}\label{netiquette}}

When using Zoom and Slack, please follow appropriate internet etiquette
(``Netiquette''). Written communications, like blog posts or use of the
Zoom chat, lacks important nonverbal cues (such as body language, tone
of voice, sarcasm, etc).

Above all else, please respect one another and think/reread carefully
about how others may see your post before you submit a comment. You are
expected to disagree and have different opinions, this is inherently
valuable in a discussion. Please be civil and constructive in responding
to others' comments: writing \emph{``have you considered `X'?''} is a
lot more helpful to all involved than just writing \emph{``well you're
just wrong.''}

Posting content that is wilfully incindiary, illegal, or that
constitutes academic dishonesty (such as plagarism) will automatically
earn a grade of 0 and may be elevated to other authorities on campus.

When using the chat function on Zoom or public Slack channels, please
treat it as official course communications, even though I may not be
grading it. It may be a quick and informal tool - don't feel you need to
worry about spelling or perfect grammar - but please try to avoid
\emph{too} informal ``text-speak'' (i.e.~say ``That's good for you''
instead of ``thas good 4 u'').

\hypertarget{privacy}{%
\subsection{Privacy}\label{privacy}}

\href{https://www.execvision.io/blog/maryland-call-recording-laws/}{Maryland
law}
\href{https://law.justia.com/codes/maryland/2005/gcj/10-402.html}{requires}
all parties consent for a conversation or meeting to be recorded. If you
join in, and certainly if you participate, \textbf{you are consenting to
be recorded.} However, as described below, videos are \emph{not
accessible} beyond our class.

Live lectures are recorded on Zoom and posted to Blackboard via Panopto,
a secure course management system for video. Among other nice features
(such as multiple video screens, close captioning, and time-stamped
search functions!), Panopto is authenticated via your Blackboard
credentials, ensuring that \emph{our course videos are not accessible to
the open internet.}

For the privacy of your peers, and to foster an environment of trust and
academic freedom to explore ideas, \textbf{do not record our course
lectures or discussions.} You are already getting my official copies.

The
\href{https://www2.ed.gov/policy/gen/guid/fpco/ferpa/index.html}{Family
Educational Rights and Privacy Act} prevents me from disclosing or
discussing any student information, including grades and records about
student performance. If the student is at least 18 years of age,
\emph{parents (or spouses) do not have a right to obtain this
information}, except with consent by the student.

Many of you may be tuning in remotely, living with parents, and may have
occasional interruptions due to sharing a space. This is normal and
fine, but know that I will protect your privacy and not discuss your
performance when parents (or anyone other than you, for that matter) are
present, without your explicit consent.

\hypertarget{enrollment}{%
\subsection{Enrollment}\label{enrollment}}

Students are responsible for verifying their enrollment in this class.
The last day to add or drop this class with no penalty is
\textbf{Wednesday, September 1}. Be aware of
\href{https://www.hood.edu/offices-services/registrars-office/academic-calendar}{important
dates}.

\hypertarget{honor-code}{%
\subsection{Honor Code}\label{honor-code}}

Hood College has an Academic Honor Code which requires all members of
this community to maintain the highest standards of academic honesty and
integrity. Cheating, plagiarism, lying, and stealing are all prohibited.
All violations of the Honor Code are taken seriously, will be reported
to appropriate authority, and may result in severe penalties, including
expulsion from the college. See
\href{http://hood.smartcatalogiq.com/en/2016-2017/Catalog/The-Spirit-of-Hood/The-Academic-Honor-Code-and-Code-of-Conduct}{here}
for more detailed information.

\hypertarget{van-halen-and-mms}{%
\subsection{Van Halen and M\&Ms}\label{van-halen-and-mms}}

When you have completed reading the syllabus, email me a picture of the
band Van Halen and a picture of a bowl of M\&Ms.~If you do this
\emph{before} the date of the first exam, you will get bonus points on
the exam. If 75-100\% of the class does this, you each get 2 points. If
50-75\% of the class does this, you each get 4 points. If 25-50\% of the
class does this, you each get 6 points. If 0-25\% of the class does
this, you each get 8 points. Yes, you read this correctly.

\hypertarget{accessibility-equity-and-accommodations}{%
\subsection{Accessibility, Equity, and
Accommodations}\label{accessibility-equity-and-accommodations}}

College courses can, and should, be challenging and bring you out of
your comfort zone in a safe and equitable environment. If, however, you
feel at any point in the semester that certain assignments or aspects of
the course will be disproportionately uncomfortable or burdensome for
you due to any factor beyond your control, please come see me or email
me. I am a very understanding person and am happy to work out a solution
together. I reserve the right to modify and reweight assignments at my
sole discretion for students that I belive would legitimately be at a
disadvantage, through no fault of their own, to complete them as
described.

If you are unable to afford required textbooks or other resources for
any reason, come see me and we can find a solution that works for you.

This course is intended to be accessible for all students, including
those with mental, physical, or cognitive disabilities, illness,
injuries, impairments, or any other condition that tends to negatively
affect one's equal access to education. If at any point in the term, you
find yourself not able to fully access the space, content, and
experience of this course, you are welcome to contact me to discuss your
specific needs. I also encourage you to contact the
\href{https://www.hood.edu/academics/josephine-steiner-center-academic-achievement-retention/accessibility-services}{Office
of Accessibility Services} (301-696-3421). If you have a diagnosis or
history of accommodations in high school or previous postsecondary
institutions, Accessibility Services can help you document your needs
and create an accommodation plan. By making a plan through Accessibility
Services, you can ensure appropriate accommodations without disclosing
your condition or diagnosis to course instructors.

\hypertarget{tentative-schedule}{%
\section{Tentative Schedule}\label{tentative-schedule}}

\textbf{You can find a full schedule} with much more details, including
the readings, appendices, and other further resources for each class
meeting on the \href{schedule/}{schedule page}.

\end{document}